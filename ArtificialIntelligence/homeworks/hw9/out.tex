\hypertarget{ai-hw9}{%
\section{AI HW9}\label{ai-hw9}}

Wang Huaqiang

2016K8009929035

\hypertarget{section}{%
\section{8.15}\label{section}}

The two sentences can not imply the fact that \(x \notin s\).

For example, prove that 1 is not in \(\emptyset\). We know from the
defination that 1 is in all the sets with a 1 in it, yet we will never
know if 1 is in \(\emptyset\) for nothing can be drived from the 2
rules.

\hypertarget{section-1}{%
\section{8.20}\label{section-1}}

Lemma:

We first define prediction ``is Nature Number'' N(x).

\[N(x):=  (x=0)\vee((x=y+1)\wedge(N(y))) \]

Also, we can generate nature number z by using function ``+'' for
multiple times:

\[z := +(1,+(1,...)) \ ((z-1)functions \ in \ total)\]

For example:

\[3:=  +(1,+(1,1)) \]

Based on these lemmas:

\hypertarget{a}{%
\subsection{8.20.a}\label{a}}

\[ iseven(x) := \exist y \ x = +(y,y)\]

or

\[ iseven(x) := \exist y (\ N(y)\wedge (<(X(2,y),x)\wedge (<(x,X(2,+(y,1))))\vee (x=0)\]

\hypertarget{b}{%
\subsection{8.20.b}\label{b}}

\[ isprime(x) := \neg \exist a,b \ x=X(a,b)\wedge(<(1,a))\wedge(<(1,b))\wedge N(a) \wedge N(b)\]

or

\[ isprime(x) := \forall a,b \ x=X(a,b)\rightarrow (a=1)\vee (b=1)\]

\hypertarget{c}{%
\subsection{8.20.c}\label{c}}

\[Goldbach := \forall x \ \exist a,b \ \ iseven(x) \wedge isprime(x) \wedge isprime(b) \wedge (x=X(a,b))\]

\sout{So in fact we do not need these lemmas. 23333}

\begin{center}\rule{0.5\linewidth}{\linethickness}\end{center}

Copyright (C) 2018 Wang Huaqiang
