\documentclass{article} % For LaTeX2e
% \usepackage{nips13submit_e,times}
\usepackage{hyperref}
\usepackage{url}
\usepackage{graphicx}
\usepackage{amsmath}
\usepackage{algorithm}
\usepackage{algorithmic}
\usepackage{ulem}
\usepackage{textcomp}  % \textcelsius


\providecommand{\tightlist}{%
	\setlength{\itemsep}{0pt}\setlength{\parskip}{0pt}}

% \graphicspath{{./fig/}}
% \usepackage{tikz}

\title{Homework 9}
\usepackage{geometry}
\geometry{left=2.5cm,right=2.5cm,top=2.5cm,bottom=2.5cm}

\author{Wang Huaqiang}

\begin{document}

	\maketitle


  \hypertarget{section}{%
  \section{8.15}\label{section}}
  
  \paragraph{}
  
  The two sentences can not imply the fact that \(x \notin s\).
  
  \paragraph{}
  
  For example, prove that 1 is not in \(\emptyset\). We know from the
  defination that 1 is in all the sets with a 1 in it, yet we will never
  know if 1 is in \(\emptyset\) for nothing can be drived from the 2
  rules.
  
  \hypertarget{section-1}{%
  \section{8.20}\label{section-1}}
  
  Lemma:
  
  ~\\
  
  We first define prediction ``is Nature Number'' N(x).
  
  \[N(x):=  (x=0)\vee((x=y+1)\wedge(N(y))) \]
  
  Also, we can generate nature number z by using function ``+'' for
  multiple times:
  
  \[z := +(1,+(1,...)) \ ((z-1)functions \ in \ total)\]
  
  For example:
  
  \[3:=  +(1,+(1,1)) \]
  
  Based on these lemmas:
  
  \hypertarget{a}{%
  \subsection{8.20.a}\label{a}}
  
  \[iseven(x) := \exists y \ x = +(y,y)\]
  
  or
  
  \[ iseven(x) := \exists y (\ N(y)\wedge (<(X(2,y),x)\wedge (<(x,X(2,+(y,1))))\vee (x=0)\]
  
  \hypertarget{b}{%
  \subsection{8.20.b}\label{b}}
  
  \[ isprime(x) := \neg \exists a,b \ x=X(a,b)\wedge(<(1,a))\wedge(<(1,b))\wedge N(a) \wedge N(b)\]
  
  or
  
  \[ isprime(x) := \forall a,b \ x=X(a,b)\rightarrow (a=1)\vee (b=1)\]
  
  \hypertarget{c}{%
  \subsection{8.20.c}\label{c}}
  
  \[Goldbach := \forall x \ \exists a,b \ \ iseven(x) \wedge isprime(x) \wedge isprime(b) \wedge (x=X(a,b))\]
  
  \sout{So in fact we do not need these lemmas. 23333}
  
  \begin{center}\rule{0.5\linewidth}{\linethickness}\end{center}
  
  \begin{center}
  Copyright (C) 2018 Wang Huaqiang 
  \end{center}

\end{document}
