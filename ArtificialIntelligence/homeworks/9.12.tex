\documentclass{article} % For LaTeX2e
% \usepackage{nips13submit_e,times}
\usepackage{hyperref}
\usepackage{url}
\usepackage{graphicx}
\usepackage{amsmath}
\usepackage{algorithm}
\usepackage{algorithmic}
\usepackage{textcomp}  % \textcelsius


% \graphicspath{{./fig/}}
% \usepackage{tikz}

\title{Homework 2}
\usepackage{geometry}
\geometry{left=2.5cm,right=2.5cm,top=2.5cm,bottom=2.5cm}



\newcommand{\fix}{\marginpar{FIX}}
\newcommand{\new}{\marginpar{NEW}}

\newtheorem{theorem}{Theorem}
\newtheorem{proof}{Proof}
\newtheorem{question}{Question}
\newtheorem{algorithm_}{Algorithm}

\begin{document}

\maketitle
%
% \begin{abstract}
% empty
% \end{abstract}

%%------------------------------------------------------------------------
% 2016K8009929035_王华强_02

\section{Homework 2}



% <!-- 最佳的k划分( -->

% <!-- 设计算法找到使得回归树的greedy function最佳的k划分(k是任意正整数). 要求先形式化这一问题, 再给出解决算法.

% 提示: 动态规划. -->

\begin{question} 
	
Best K-division

Let N be a node of a regression tree, design a greedy algorithm to find the best division of the data set, which ensures the greatest target function loss.

The target function is defined as:

$$\sum_{j=1}^{T}(\frac{-2B_j^2}{2A_j+\lambda})+\gamma T+Const$$

Where $I_i$ represents a node, $y_i$ is the real value of a record, T is the number of leaf nodes generated, $A_j=|I_j|$, $B_j=\sum_{j}y_j$.

We assume that the number of elements of the node is $NodeSize$.

\end{question}

\begin{algorithm_}
	
Faster Enumeration

Firstly we fix the sequence of all the elements.

A basic idea is to enumerate all posible ways of division. By dynamic programming, we can lower the cost.

We use 2-d array $Result$ to record intermediate results. $Result[j][i]$ means the lowest target function value from $division(i,j)$, which means the last division happend before the gap between element `i` and `i+1` with j divisions in total.

Thus we can get the algorithm represented by pseudocode.

\begin{algorithm}
	\caption{Faster Enumeration}         %单独一栏标题
	\label{alg1}                          %标签显示为:Algorithm 1
		

\begin{verbatim}
//Prepare Data
Let Result[<0][]=INF;
Let Result[][<0]=INF;
Let Result[>=NodeSize][]=INF;
Let Result[][>=NodeSize]=INF;

//Dynamic Programming
for(j=0;j<NodeSize;j++)
    for(i=0;i<NodeSize;i++)
    {
        Result[j][i]=
            min{Result[j][i-1],(Result[j-1][i]+Loss(j,i))}; 
        Update recordofdivisions[j][i];//record the position of divisions.  
    };

//Pick Result
Return min(Result), as well as the way of division;

\end{verbatim}
\end{algorithm}

`Loss(j,i)` is the loss gain from adding a division at the ith gap. It is related to `recordofdivisions[j-1][i]`.

\end{algorithm_}

\end{document}
