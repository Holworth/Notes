\PassOptionsToPackage{unicode=true}{hyperref} % options for packages loaded elsewhere
\PassOptionsToPackage{hyphens}{url}
%
\documentclass[]{article}
\usepackage{lmodern}
\usepackage{amssymb,amsmath}
\usepackage{ifxetex,ifluatex}
\usepackage{fixltx2e} % provides \textsubscript
\ifnum 0\ifxetex 1\fi\ifluatex 1\fi=0 % if pdftex
  \usepackage[T1]{fontenc}
  \usepackage[utf8]{inputenc}
  \usepackage{textcomp} % provides euro and other symbols
\else % if luatex or xelatex
  \usepackage{unicode-math}
  \defaultfontfeatures{Ligatures=TeX,Scale=MatchLowercase}
\fi
% use upquote if available, for straight quotes in verbatim environments
\IfFileExists{upquote.sty}{\usepackage{upquote}}{}
% use microtype if available
\IfFileExists{microtype.sty}{%
\usepackage[]{microtype}
\UseMicrotypeSet[protrusion]{basicmath} % disable protrusion for tt fonts
}{}
\IfFileExists{parskip.sty}{%
\usepackage{parskip}
}{% else
\setlength{\parindent}{0pt}
\setlength{\parskip}{6pt plus 2pt minus 1pt}
}
\usepackage{hyperref}
\hypersetup{
            pdfborder={0 0 0},
            breaklinks=true}
\urlstyle{same}  % don't use monospace font for urls
\usepackage{color}
\usepackage{fancyvrb}
\newcommand{\VerbBar}{|}
\newcommand{\VERB}{\Verb[commandchars=\\\{\}]}
\DefineVerbatimEnvironment{Highlighting}{Verbatim}{commandchars=\\\{\}}
% Add ',fontsize=\small' for more characters per line
\newenvironment{Shaded}{}{}
\newcommand{\AlertTok}[1]{\textcolor[rgb]{1.00,0.00,0.00}{\textbf{#1}}}
\newcommand{\AnnotationTok}[1]{\textcolor[rgb]{0.38,0.63,0.69}{\textbf{\textit{#1}}}}
\newcommand{\AttributeTok}[1]{\textcolor[rgb]{0.49,0.56,0.16}{#1}}
\newcommand{\BaseNTok}[1]{\textcolor[rgb]{0.25,0.63,0.44}{#1}}
\newcommand{\BuiltInTok}[1]{#1}
\newcommand{\CharTok}[1]{\textcolor[rgb]{0.25,0.44,0.63}{#1}}
\newcommand{\CommentTok}[1]{\textcolor[rgb]{0.38,0.63,0.69}{\textit{#1}}}
\newcommand{\CommentVarTok}[1]{\textcolor[rgb]{0.38,0.63,0.69}{\textbf{\textit{#1}}}}
\newcommand{\ConstantTok}[1]{\textcolor[rgb]{0.53,0.00,0.00}{#1}}
\newcommand{\ControlFlowTok}[1]{\textcolor[rgb]{0.00,0.44,0.13}{\textbf{#1}}}
\newcommand{\DataTypeTok}[1]{\textcolor[rgb]{0.56,0.13,0.00}{#1}}
\newcommand{\DecValTok}[1]{\textcolor[rgb]{0.25,0.63,0.44}{#1}}
\newcommand{\DocumentationTok}[1]{\textcolor[rgb]{0.73,0.13,0.13}{\textit{#1}}}
\newcommand{\ErrorTok}[1]{\textcolor[rgb]{1.00,0.00,0.00}{\textbf{#1}}}
\newcommand{\ExtensionTok}[1]{#1}
\newcommand{\FloatTok}[1]{\textcolor[rgb]{0.25,0.63,0.44}{#1}}
\newcommand{\FunctionTok}[1]{\textcolor[rgb]{0.02,0.16,0.49}{#1}}
\newcommand{\ImportTok}[1]{#1}
\newcommand{\InformationTok}[1]{\textcolor[rgb]{0.38,0.63,0.69}{\textbf{\textit{#1}}}}
\newcommand{\KeywordTok}[1]{\textcolor[rgb]{0.00,0.44,0.13}{\textbf{#1}}}
\newcommand{\NormalTok}[1]{#1}
\newcommand{\OperatorTok}[1]{\textcolor[rgb]{0.40,0.40,0.40}{#1}}
\newcommand{\OtherTok}[1]{\textcolor[rgb]{0.00,0.44,0.13}{#1}}
\newcommand{\PreprocessorTok}[1]{\textcolor[rgb]{0.74,0.48,0.00}{#1}}
\newcommand{\RegionMarkerTok}[1]{#1}
\newcommand{\SpecialCharTok}[1]{\textcolor[rgb]{0.25,0.44,0.63}{#1}}
\newcommand{\SpecialStringTok}[1]{\textcolor[rgb]{0.73,0.40,0.53}{#1}}
\newcommand{\StringTok}[1]{\textcolor[rgb]{0.25,0.44,0.63}{#1}}
\newcommand{\VariableTok}[1]{\textcolor[rgb]{0.10,0.09,0.49}{#1}}
\newcommand{\VerbatimStringTok}[1]{\textcolor[rgb]{0.25,0.44,0.63}{#1}}
\newcommand{\WarningTok}[1]{\textcolor[rgb]{0.38,0.63,0.69}{\textbf{\textit{#1}}}}
\setlength{\emergencystretch}{3em}  % prevent overfull lines
\providecommand{\tightlist}{%
  \setlength{\itemsep}{0pt}\setlength{\parskip}{0pt}}
\setcounter{secnumdepth}{0}
% Redefines (sub)paragraphs to behave more like sections
\ifx\paragraph\undefined\else
\let\oldparagraph\paragraph
\renewcommand{\paragraph}[1]{\oldparagraph{#1}\mbox{}}
\fi
\ifx\subparagraph\undefined\else
\let\oldsubparagraph\subparagraph
\renewcommand{\subparagraph}[1]{\oldsubparagraph{#1}\mbox{}}
\fi

% set default figure placement to htbp
\makeatletter
\def\fps@figure{htbp}
\makeatother


\date{}

\begin{document}

\hypertarget{ux4ebaux5de5ux667aux80fdux57faux7840ux4f5cux4e1a-9.12}{%
\section{人工智能基础作业
9.12}\label{ux4ebaux5de5ux667aux80fdux57faux7840ux4f5cux4e1a-9.12}}

\begin{itemize}
\tightlist
\item
  王华强
\item
  2016K8009929035
\end{itemize}

\begin{center}\rule{0.5\linewidth}{\linethickness}\end{center}

\hypertarget{ux6700ux4f73ux7684kux5212ux5206best-k-division}{%
\subsection{最佳的k划分(Best
K-division)}\label{ux6700ux4f73ux7684kux5212ux5206best-k-division}}

设计算法找到使得回归树的greedy function最佳的k划分(k是任意正整数).
要求先形式化这一问题, 再给出解决算法.

提示: 动态规划.

\hypertarget{question}{%
\subsection{Question}\label{question}}

Let N be a node of a regression tree, design a greedy algorithm to find
the best division of the data set, which ensures the greatest target
function loss.

The target function is defined as:

\[\sum_{j=1}^{T}(\frac{-2B_j^2}{2A_j+\lambda})+\gamma T+Const\]

Where \(I_i\) represents a node, \(y_i\) is the real value of a record,
T is the number of leaf nodes generated, \(A_j=|I_j|\),
\(B_j=\sum_{j}y_j\).

We assume that the number of elements of the node is
\texttt{Node\_Size}.

\hypertarget{algorithm}{%
\subsection{Algorithm}\label{algorithm}}

Firstly we fix the sequence of all the elements.

A basic idea is to enumerate all posible ways of division. By dynamic
programming, we can lower the cost.

We use 2-d array \texttt{Result} to record intermediate results.
\texttt{Result{[}j{]}{[}i{]}} means the lowest target function value
from \texttt{division(i,j)}, which means the last division happend
before the gap between element \texttt{i} and \texttt{i+1} with j
divisions in total.

Thus we can get the algorithm represented by pseudocode.

\begin{Shaded}
\begin{Highlighting}[]
\CommentTok{//Prepare Data}
\NormalTok{Let Result[<}\DecValTok{0}\NormalTok{][]=INF;}
\NormalTok{Let Result[][<}\DecValTok{0}\NormalTok{]=INF;}
\NormalTok{Let Result[>=Node_Size][]=INF;}
\NormalTok{Let Result[][>=Node_Size]=INF;}

\CommentTok{//Dynamic Programming}
\ControlFlowTok{for}\NormalTok{(j=}\DecValTok{0}\NormalTok{;j<Node_Size;j++)}
    \ControlFlowTok{for}\NormalTok{(i=}\DecValTok{0}\NormalTok{;i<Node_Size;i++)}
\NormalTok{    \{}
\NormalTok{        Result[j][i]=}
\NormalTok{            min\{Result[j][i}\DecValTok{-1}\NormalTok{],(Result[j}\DecValTok{-1}\NormalTok{][i]+Loss(j,i))\};}
\NormalTok{        Update record_of_divisions[j][i];}\CommentTok{//record the position of divisions.}
\NormalTok{    \};}

\CommentTok{//Pick Result}
\NormalTok{Return min(Result), as well as the way of division;}
\end{Highlighting}
\end{Shaded}

Loss(j,i) is the loss gain from adding a division at the ith gap.

\end{document}
