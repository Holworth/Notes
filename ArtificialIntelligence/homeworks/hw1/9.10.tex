\documentclass{article} % For LaTeX2e
% \usepackage{nips13submit_e,times}
\usepackage{hyperref}
\usepackage{url}
\usepackage{graphicx}
\usepackage{amsmath}
\usepackage{algorithm}
\usepackage{algorithmic}
\usepackage{textcomp}  % \textcelsius
% \graphicspath{{./fig/}}
% \usepackage{tikz}

\title{Homework 1}
\usepackage{geometry}
\geometry{left=2.5cm,right=2.5cm,top=2.5cm,bottom=2.5cm}



\newcommand{\fix}{\marginpar{FIX}}
\newcommand{\new}{\marginpar{NEW}}

\newtheorem{theorem}{Theorem}
\newtheorem{proof}{Proof}

\begin{document}


\maketitle
%
% \begin{abstract}
% empty
% \end{abstract}

%%------------------------------------------------------------------------
% 2016K8009929035_王华强_01A

\section{Homework 1A}

% 作业A:决策树-PPT 11页

%     形式化定义:条件信息熵

%     进行证明:信息增益一定大于等于0

% 作业B:决策树-PPT 12页

%     决策树:条件信息熵的应用

%     构造一个数据,每一个条件数据有自己的取值,

%     构造的数据要体现用greedy算出来的决策树不是一个最优的决策树
    

We define:

$$Gain(X,Y)=H(Y)-H(Y|X)$$
$$H(Y)=-\sum_{j}^{}P(I_j)log(P(I_j))$$
$$H(Y|X)=\sum_{i}H(Y|x_i)P(x_i)$$

Then the theorem can be formally defined as:

$$Gain(X,Y)=H(Y)-H(Y|X)>=0$$

\begin{theorem}

    $Gain(X,Y)=H(Y)-H(Y|X)>=0$

\end{theorem}

\begin{proof}


From the defination, we gain:

$$Gain(X,Y)=H(Y)-H(Y|X)$$
$$=-\sum_{j}^{}P(I_j)log(P(I_j))-\sum_{i}H(Y|x_i)P(x_i)$$
$$=-\sum_{j}^{}P(I_j)log(P(I_j))+\sum_{i}(\sum_{j}^{}P(I_j|x_i)log(P(I_j|x_i))))P(x_i)$$
$$=-\sum_{j}^{}log(P(I_j))\sum_{i}P(I_j|x_i)P(x_i)+\sum_{i}P(x_i)\sum_{j}^{}P(I_j|x_i)log(P(I_j|x_i))$$
$$=\sum_{i,j}P(I_j|x_i)P(x_i)log(\frac{P(I_j|x_i)}{P(I_j)})$$
$$=\sum_{i,j}P(I_jx_i)log(\frac{P(I_jx_i)}{P(I_j)P(x_i)})$$
$$=\sum_{i,j}P(I_jx_i)-log(\frac{P(I_j)P(x_i)}{P(I_jx_i)})$$

-log is a convex function, then we can use Jensen unequation.

From Jensen unequation:

$$\sum_{i,j}P(I_jx_i)-log(\frac{P(I_j)P(x_i)}{P(I_jx_i)})$$
$$>=\sum_{i,j}-log(P(I_jx_i)\frac{P(I_j)P(x_i)}{P(I_jx_i)})$$
$$=\sum_{i,j}-log(P(I_j)P(x_i))$$
$$=-log(1)=0$$

Therefore:

$$Gain(X,Y)=H(Y)-H(Y|X)>=0$$

Also, we may import KL divergence, and prove that KL divergence is non-negative;


ref: https://blog.csdn.net/MathThinker/article/details/48375523

\end{proof}

% \section{as2}

% ttttttt

\end{document}
