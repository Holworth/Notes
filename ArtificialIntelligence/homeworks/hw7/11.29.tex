\documentclass{article} % For LaTeX2e
% \usepackage{nips13submit_e,times}
\usepackage{hyperref}
\usepackage{url}
\usepackage{graphicx}
\usepackage{amsmath}
\usepackage{algorithm}
\usepackage{algorithmic}
\usepackage{textcomp}  % \textcelsius


\providecommand{\tightlist}{%
	\setlength{\itemsep}{0pt}\setlength{\parskip}{0pt}}

% \graphicspath{{./fig/}}
% \usepackage{tikz}

\title{Homework 7}
\usepackage{geometry}
\geometry{left=2.5cm,right=2.5cm,top=2.5cm,bottom=2.5cm}

\author{Wang Huaqiang}

\begin{document}

	\maketitle



\begin{Huge}
	\begin{center}
		Contents
	\end{center}
\end{Huge}


\begin{itemize}
\tightlist
\item
  \protect\hyperlink{1-question}{1. Question}
\item
  \protect\hyperlink{2-lemma}{2. Lemma}

  \begin{itemize}
  \tightlist
  \item
    \protect\hyperlink{21-belong-to-ux2fin-in}{2.1. belong to (\(\in\))
    (in)}
  \item
    \protect\hyperlink{22-neg-imply}{2.2. neg imply}
  \item
    \protect\hyperlink{23-transitivity1-trans}{2.3. transitivity1
    (trans)}
  \item
    \protect\hyperlink{24-transitivity2-tr}{2.4. transitivity2 (Tr)}
  \item
    \protect\hyperlink{25-double-neg}{2.5. double neg}
  \item
    \protect\hyperlink{26-reductio}{2.6. reductio}
  \item
    \protect\hyperlink{27-confliction}{2.7. confliction}
  \item
    \protect\hyperlink{28-and}{2.8. and}
  \item
    \protect\hyperlink{29-switch}{2.9. switch}
  \end{itemize}
\item
  \protect\hyperlink{3-proof}{3. Proof}

  \begin{itemize}
  \tightlist
  \item
    \protect\hyperlink{31-t1}{3.1. T1}
  \item
    \protect\hyperlink{32-t2}{3.2. T2}
  \item
    \protect\hyperlink{33-t3}{3.3. T3}
  \item
    \protect\hyperlink{34-t4}{3.4. T4}
  \item
    \protect\hyperlink{35-t5}{3.5. T5}
  \item
    \protect\hyperlink{36-t6}{3.6. T6}
  \item
    \protect\hyperlink{37-t7}{3.7. T7}
  \item
    \protect\hyperlink{38-t8}{3.8. T8}
  \end{itemize}
\end{itemize}


\par

\begin{Huge}
	\begin{center}
		Remarks
	\end{center}
\end{Huge}


Made a mistake while trying to convert markdown to tex. So there are some wrong feedlines for formulas.

For example:

\[A\vdash A \vee B, B \vee A \] (T1)

Means:

\[A\vdash A \vee B, B \vee A \quad \quad(T1)\] 

\newpage

\hypertarget{question}{%
\section{Question}\label{question}}

Prove that:

\[A\vdash A \vee B, B \vee A \] (T1)

\[A\vee B \vdash \dashv B \vee A\] (T2)

\[(A \vee B)\vee C \vdash \dashv A \vee (B \vee C)\] (T3)

\[(A \vee B) \vdash \dashv \neg A \rightarrow B\] (T4)

\[ A \rightarrow B \vdash \dashv \neg A \vee B\] (T5)

\[\neg (A \vee B) \vdash \dashv \neg A \wedge \neg B\] (T6)

\[\neg (A \wedge B) \vdash \dashv \neg A \vee \neg B\] (T7)

\[\emptyset \vdash A \vee \neg A\] (T8)

\hypertarget{lemma}{%
\section{Lemma}\label{lemma}}

First we prove the following lemmas to simpilify the proof.

\hypertarget{belong-to-in-in}{%
\subsection{\texorpdfstring{2.1. belong to (\(\in\))
(in)}{2.1. belong to (\textbackslash{}in) (in)}}\label{belong-to-in-in}}

If:

\[A \in \Sigma\]

Then:

\[\Sigma\vdash A\]

Proof:

Let \(\Sigma'\) be \(\Sigma - A\), then:

\[A\vdash A\](Ref) \[\Sigma', A\vdash A\](+) \[\Sigma \vdash A\] (1,2)

\hypertarget{neg-imply}{%
\subsection{2.2. neg imply}\label{neg-imply}}

\[\neg A \rightarrow B \vdash \neg B \rightarrow A\]

proof:

\[\neg A \rightarrow B , \neg A, \neg B\vdash \neg A \rightarrow B \](in)
\[\neg A \rightarrow B , \neg A, \neg B\vdash \neg A\](in)

From 1, 2, \(\rightarrow -\):

\[\neg A \rightarrow B , \neg A, \vdash B\]

\[\neg A \rightarrow B , \neg A, \neg B\vdash B\](3)
\[\neg A \rightarrow B , \neg A, \neg B\vdash \neg B\](in)

From \(\neg -\), 4 ,5 :

\[\neg A \rightarrow B , \neg B\vdash A\]

From 6, \(rightarrow +\):

\[\neg A \rightarrow B \vdash \neg B \rightarrow A\]

\hypertarget{transitivity1-trans}{%
\subsection{2.3. transitivity1 (trans)}\label{transitivity1-trans}}

\[A\rightarrow B, B\rightarrow C \vdash A\rightarrow C\] (trans)

Proof:

\[A\rightarrow B, B\rightarrow C , A \vdash A\]
\[A\rightarrow B, B\rightarrow C , A \vdash A\rightarrow B\]
\[A\rightarrow B, B\rightarrow C , A \vdash B\]
\[A\rightarrow B, B\rightarrow C , A \vdash B\rightarrow C\]
\[A\rightarrow B, B\rightarrow C , A \vdash C\]
\[A\rightarrow B, B\rightarrow C \vdash A \rightarrow C\]

\hypertarget{transitivity2-tr}{%
\subsection{2.4. transitivity2 (Tr)}\label{transitivity2-tr}}

IF:

\[\Sigma \vdash \Sigma'\] \[\Sigma' \vdash A\]

Then:

\[\Sigma \vdash A\] (Tr)

\hypertarget{double-neg}{%
\subsection{2.5. double neg}\label{double-neg}}

\[\neg \neg A \vdash A\](2neg) \[A \vdash \neg \neg A\](2neg)

Proof1:

\[\neg \neg A , \neg A\vdash A\] \[\neg \neg A , \neg A\vdash \neg A\]

From 1, 2, \(\neg-\):

\[\neg \neg A \vdash A\]

Proof2:

\[\neg \neg \neg A \vdash \neg A\](2neg)
\[A, \neg \neg \neg A \vdash \neg A\](+)
\[A, \neg \neg \neg A \vdash A\](in)

From \(\neg -\):

\[A \vdash \neg \neg A\]

\hypertarget{reductio}{%
\subsection{2.6. reductio}\label{reductio}}

If:

\[\Sigma , A \vdash B\] \[\Sigma , A \vdash \neg B\]

Then:

\[\Sigma \vdash \neg A\] (redct)

Proof:

\[\Sigma ,A\vdash B\] \[\neg \neg A \vdash A\]

From 1, 2, Tr:

\[\Sigma \neg \neg A\vdash B\]

Similarily:

\[\neg \neg A \vdash A\] \[\Sigma ,A\vdash \neg B\]
\[\Sigma \neg \neg A\vdash \neg B\]

Finally, from 3, 6, \(\neg-\):

\[\Sigma \vdash \neg A\]

\hypertarget{confliction}{%
\subsection{2.7. confliction}\label{confliction}}

\[A, \neg A \vdash B\] (conf) \[A \vdash \neg A  \rightarrow B\] (conf)
\[\neg A \vdash A  \rightarrow B\] (conf)

Proof:

\[A, \neg A , \neg B\vdash A \] (in)
\[A, \neg A , \neg B\vdash \neg A \] (in) \[A, \neg A \vdash B\] (1,2)

From 3, \(\rightarrow+\):

\[A \vdash \neg A  \rightarrow B\] \[\neg A \vdash A  \rightarrow B\]

\hypertarget{and}{%
\subsection{2.8. and}\label{and}}

\[A\wedge B\vdash \dashv A,B\]

Proof:

\[A\wedge B \vdash A\wedge B\] (in)

From 1, \(\wedge-\):

\[A\wedge B \vdash A\] \[A\wedge B \vdash B\] \[A\wedge B \vdash A, B\]

And:

\[A, B\vdash A\] \[A, B\vdash B\]

Then use \(\wedge+\) rule;

\[A, B\vdash A\wedge B\]

\hypertarget{switch}{%
\subsection{2.9. switch}\label{switch}}

If:

\[A\vdash \dashv A'\]

\[\vdash A\rightarrow B\] \[\vdash C\rightarrow A\] \[\vdash A\vee D\]
\[\vdash A\wedge E\] \[\vdash \neg A\]

Then it is trival that:

\[\vdash A'\rightarrow B\] \[\vdash C\rightarrow A'\]
\[\vdash A'\vee D\] \[\vdash A'\wedge E\] \[\vdash \neg A'\]

Therefore, for A, A' in deduction formula, we can switch A and A'.

\hypertarget{proof}{%
\section{Proof}\label{proof}}

\hypertarget{t1}{%
\subsection{3.1. T1}\label{t1}}

\[A \vdash A\] (Ref)

\[A \vdash A \vee B\] (1,V+)

\[A \vdash B \vee A\] (1,V+)

\[A\vdash A \vee B, B \vee A \] (3,4)

\hypertarget{t2}{%
\subsection{3.2. T2}\label{t2}}

To prove:

\[A\vee B \vdash \dashv B \vee A\] (T2)

We only need to prove one direction, for this formula is symmatric.

\[A \vdash B \vee A\] (T1)

\[B \vdash B \vee A\] (T1)

From 1,2,V-:

\[A \vee B \vdash B \vee A \]

Symmatricly:

\[B \vee A \vdash A \vee B\]

As a result:

\[A\vee B \vdash \dashv B \vee A\] (4,5)

\hypertarget{t3}{%
\subsection{3.3. T3}\label{t3}}

\[C \vdash (B \vee C)\] (T1) \[(B \vee C) \vdash A \vee (B \vee C)\]
(T1) \[C\vdash A \vee (B \vee C)\] (1,2)

Similarily, we have:

\[A\vdash A \vee (B \vee C)\] \[B\vdash A \vee (B \vee C)\]

Then from 4,5,V-:

\[A\vee B \vdash A \vee (B \vee C)\]

Finally:

\[(A \vee B)\vee C \vdash A \vee (B \vee C)\] (3,6)

\hypertarget{t4}{%
\subsection{3.4. T4}\label{t4}}

Left to Right:

\[A ,\neg A \vdash B\] (conf) \[A \vdash \neg A \rightarrow B\] (conf)
\[B ,\neg A \vdash B\] (in)

From 3, \(\rightarrow+\):

\[B \vdash \neg A \rightarrow B\]

From 3, 4, V-:

\[(A \vee B) \vdash \neg A \rightarrow B\]

Right to Left:

\[A\vdash A \vee B\] (V+)

From 1, lemma neg imply:

\[\neg (A \vee B) \vdash \neg A\]

\[\neg A \rightarrow B , \neg (A \vee B)\vdash \neg A \] (+)
\[\neg A \rightarrow B , \neg (A \vee B)\vdash \neg A \rightarrow B \]
(in)

From \(\rightarrow-\):

\[\neg A \rightarrow B , \neg (A \vee B)\vdash B \]

Using V+:

\[\neg A \rightarrow B , \neg (A \vee B)\vdash (A \vee B) \]

Using in:

\[\neg A \rightarrow B , \neg (A \vee B)\vdash \neg (A \vee B) \] (in)

From \(\neg-\):

\[\neg A \rightarrow B \vdash (A \vee B) \]

To sum up:

\[(A \vee B) \vdash \dashv \neg A \rightarrow B\]

\hypertarget{t5}{%
\subsection{3.5. T5}\label{t5}}

\[(\neg A \vee B) \vdash \dashv \neg(\neg A) \rightarrow B\] (T4)

\[\neg \neg A \vdash \dashv A\]

From 1, 2, switch:

\[ \neg A \vee B \vdash \dashv A \rightarrow B \]

\hypertarget{t6}{%
\subsection{3.6. T6}\label{t6}}

Consider that:

\[\neg(\neg A \rightarrow B) , A\vdash \neg(\neg A \rightarrow B)\] (in)

From :

\[A, \neg A \vdash B\]

Using \(\rightarrow+\), we gain:

\[\neg(\neg A \rightarrow B) , A\vdash (\neg A \rightarrow B)\]

Then use \(\neg-\):

\[\neg(\neg A \rightarrow B) \vdash \neg A\]

\[\neg(\neg A \rightarrow B) \vdash \neg B\]

Since:

\[\neg(\neg A \rightarrow B) \vdash \neg A\]

\[\neg(\neg A \rightarrow B) \vdash \neg B\]

Then:

\[\neg(\neg A \rightarrow B) \vdash \neg A \wedge \neg B\]

And:

\[\neg (\neg A \rightarrow B)\vdash \neg(\neg A \rightarrow B)\] (in)

From T4, switch:

\[\neg (A \vee B)\vdash \neg(\neg A \rightarrow B)\]

\[\neg (A \vee B) \vdash \neg A \wedge \neg B\] (Tr)

Another direction:

\[\neg A \wedge \neg B \vdash \neg A, \neg B \] (and)
\[\neg A ,\neg B , (A \vee B) \vdash \neg B \] (in)
\[\neg A ,\neg B , (A \vee B) \vdash \neg A \] (in)
\[\neg A ,\neg B , (A \vee B) \vdash \neg A \rightarrow B\] (T4)

From \(\rightarrow-\):

\[\neg A ,\neg B , (A \vee B) \vdash B\]

Then from \(\neg -\):

\[\neg A ,\neg B \vdash \neg (A \vee B) \]

Using Tr:

\[\neg A \wedge \neg B \vdash \neg (A \vee B) \]

As a result:

\[\neg (A \vee B) \vdash \dashv \neg A \wedge \neg B\]

\hypertarget{t7}{%
\subsection{3.7. T7}\label{t7}}

\[\neg (A \wedge B) \vdash A \rightarrow \neg B\]

And:

\[ \neg A \vee \neg B \vdash \dashv \neg \neg A \rightarrow \neg B\]
(T4)

From 2neg, switch:

\[ \neg A \vee \neg B \rightarrow \vdash \dashv A \rightarrow \neg B\]

As a result (Tr):

\[\neg (A \wedge B) \vdash \neg A \vee \neg B\]

Another direction:

We have proved that:

\[ \neg A \vee \neg B \rightarrow \vdash \dashv A \rightarrow \neg B\]

For (we have proved it many times):

\[ A \rightarrow \neg B, A \vdash \neg B\]

Using Tr, and:

\[ \neg A \vee \neg B , (A \wedge B)\vdash \neg B \]

From in, and:

\[ \neg A \vee \neg B , (A \wedge B)\vdash B \]

Using \(\neg-\):

\[ \neg A \vee \neg B \vdash \neg (A \wedge B) \]

Both 2 directions have been proved:

\[\neg (A \wedge B) \vdash \dashv \neg A \vee \neg B\] (T7)

\hypertarget{t8}{%
\subsection{3.8. T8}\label{t8}}

\[\neg A \vdash \neg A\] (in)

From \(\rightarrow+\):

\[\emptyset \vdash \neg A \rightarrow \neg A\]

From T4:

\[\neg A \rightarrow \neg A \vdash A \vee \neg A\]

As a result:

\[\emptyset \vdash A \vee \neg A\] (1,3,Tr)

\end{document}
