\hypertarget{homework-of-introduction-of-ai}{%
\section{Homework of Introduction of
AI}\label{homework-of-introduction-of-ai}}

\begin{itemize}
\tightlist
\item
  wanghuaqiang
\item
  2016K8009929035
\item
  9.12
\end{itemize}

\begin{center}\rule{0.5\linewidth}{\linethickness}\end{center}

\hypertarget{best-k-division}{%
\subsection{Best K-division}\label{best-k-division}}

\hypertarget{question}{%
\subsection{Question}\label{question}}

Let N be a node of a regression tree, design a greedy algorithm to find
the best division of the data set, which ensures the greatest target
function loss.

The target function is defined as:

\[\sum_{j=1}^{T}(\frac{-2B_j^2}{2A_j+\lambda})+\gamma T+Const\]

Where \(I_i\) represents a node, \(y_i\) is the real value of a record,
T is the number of leaf nodes generated, \(A_j=|I_j|\),
\(B_j=\sum_{j}y_j\).

We assume that the number of elements of the node is
\texttt{Node\_Size}.

\hypertarget{algorithm}{%
\subsection{Algorithm}\label{algorithm}}

Firstly we fix the sequence of all the elements.

A basic idea is to enumerate all posible ways of division. By dynamic
programming, we can lower the cost.

We use 2-d array \texttt{Result} to record intermediate results.
\texttt{Result{[}j{]}{[}i{]}} means the lowest target function value
from \texttt{division(i,j)}, which means the last division happend
before the gap between element \texttt{i} and \texttt{i+1} with j
divisions in total.

Thus we can get the algorithm represented by pseudocode.

\begin{Shaded}
\begin{Highlighting}[]
\CommentTok{//Prepare Data}
\NormalTok{Let Result[<}\DecValTok{0}\NormalTok{][]=INF;}
\NormalTok{Let Result[][<}\DecValTok{0}\NormalTok{]=INF;}
\NormalTok{Let Result[>=Node_Size][]=INF;}
\NormalTok{Let Result[][>=Node_Size]=INF;}

\CommentTok{//Dynamic Programming}
\ControlFlowTok{for}\NormalTok{(j=}\DecValTok{0}\NormalTok{;j<Node_Size;j++)}
    \ControlFlowTok{for}\NormalTok{(i=}\DecValTok{0}\NormalTok{;i<Node_Size;i++)}
\NormalTok{    \{}
\NormalTok{        Result[j][i]=}
\NormalTok{            min\{Result[j][i}\DecValTok{-1}\NormalTok{],(Result[j}\DecValTok{-1}\NormalTok{][i]+Loss(j,i))\};}
\NormalTok{        Update record_of_divisions[j][i];}\CommentTok{//record the position of divisions.}
\NormalTok{    \};}

\CommentTok{//Pick Result}
\NormalTok{Return min(Result), as well as the way of division;}
\end{Highlighting}
\end{Shaded}

\texttt{Loss(j,i)} is the loss gain from adding a division at the ith
gap. It is related to \texttt{record\_of\_divisions{[}j-1{]}{[}i{]}}.
