\documentclass{article} % For LaTeX2e
% \usepackage{nips13submit_e,times}
\usepackage{hyperref}
\usepackage{url}
\usepackage{graphicx}
\usepackage{amsmath}
\usepackage{algorithm}
\usepackage{algorithmic}
\usepackage{textcomp}  % \textcelsius


\providecommand{\tightlist}{%
	\setlength{\itemsep}{0pt}\setlength{\parskip}{0pt}}

% \graphicspath{{./fig/}}
% \usepackage{tikz}

\title{Homework 7}
\usepackage{geometry}
\geometry{left=2.5cm,right=2.5cm,top=2.5cm,bottom=2.5cm}

\author{Wang Huaqiang}

\begin{document}
	
	\maketitle
	
	
	
	\hypertarget{a}{%
		\section{a}\label{a}}
	
	Proof:
	
	To prove:
	
	\[(A \vee B ) \wedge (\neg A \vee C) \wedge (\neg B \vee D) \wedge (\neg C \vee G ) \wedge (\neg D \vee G) \ entails \  G\]
	
	We include \(\neg G\) in this sentence:
	
	\[(A \vee B ) \wedge (\neg A \vee C) \wedge (\neg B \vee D) \wedge (\neg C \vee G ) \wedge (\neg D \vee G) \wedge \neg G\]
	
	By using Resolution Rule:
	
	\[(A \vee B ) \wedge (\neg A \vee C) \ generates \ (B \vee C) \]
	\[(B \vee C) \wedge (\neg B \vee D) \ generates \ (C \vee D) \]
	\[(C \vee D) \wedge (\neg C \vee G) \ generates \ (D \vee G) \]
	\[(D \vee G) \wedge (\neg D \vee G) \ generates \ (G) \]
	\[(G) \wedge (\neg G) \ generates \ \emptyset  \]
	
	Hence:
	
	\[(A \vee B ) \wedge (\neg A \vee C) \wedge (\neg B \vee D) \wedge (\neg C \vee G ) \wedge (\neg D \vee G) \ entails \  G\]
	
	Q.E.D.
	
	\hypertarget{b}{%
		\section{b}\label{b}}
	
	For n proposition symbols:
	
	Consider the clauses as following:
	
	\((A \vee \neg A)\), \((B \vee \neg B)\), \ldots{} (n in total)
	
	These clauses share the same semantic: Tautology (1)
	
	For other clauses, such as \((A \vee \neg B)\) and \((B \vee \neg C)\),
	they are different in semantic in pairs.
	
	Also, consider clauses such as \((A \vee A)\). There are n such clauses.
	
	To avoid repeated count caused by commutative law, just use number of
	combination to count. As all symbols also have a negative form, the
	total number of symbols is 2n. Therefore, the result is:
	
	\[\binom{2n}{2} - n + n + 1\]
	
	i.e.
	
	\[\binom{2n}{2} + 1\]
	
	\hypertarget{c}{%
		\section{c}\label{c}}
	
	Combine 2-CNF clauses with the same semantic takes polynomial time. In
	each iteration in propositional resolution, 2 2-CNF clauses generate 1
	new clause in polynomial time. As there are at most
	\(\binom{2n}{2} + 1\) different clauses, iteration can last at most
	\(\binom{2n}{2} + 1\) rounds, and each round needs polynomial time, the
	total time is at most:
	
	\[polytime*polytime+polytime\]
	
	Which is still polynomial time.
	
	\hypertarget{d}{%
		\section{d}\label{d}}
	
	In the proof above-mentioned, ``2 2-CNF clauses generate 1 new clause in
	polynomial time'' is a critical condition. Yet for 3-CNFs, resolute 2
	3-CNF will not necessarily generate 3-CNF or 2-CNF.
	
	\begin{center}\rule{0.5\linewidth}{\linethickness}\end{center}
	
	Copyright (C) 2018 Wang Huaqiang
	
	
\end{document}
